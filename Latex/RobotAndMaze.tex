\documentclass[12pt]{article}

\usepackage{graphicx} % to include figures
%\graphicspath{{Figures/}} %This can be used if figures are in a folder in the Latex path
%\graphicspath{{../Matlab/MainProject/Figures/}} %Setting the graphics path
\usepackage{float} % to force figure placement with [H]
\usepackage{geometry} % to set the page size and margins
\geometry{a4paper, margin=2.5cm}
\usepackage{amsmath,amsthm,amssymb} % enables subequations and aligned
\usepackage{matlab-prettifier} % to import MATLAB code
\usepackage[hidelinks]{hyperref} % enable hyperlinks (boxes hidden)
\usepackage{pgf} % enable pgf figures
\usepackage{siunitx} % enable SI units and notation
\usepackage{arydshln} % for partitioned matrices with dashed lines

\usepackage{datetime2}

\usepackage{esint}
\usepackage{pxfonts}

\usepackage{tkz-euclide}

%% packages for pfg plots from MATLAB's exported tex files
%% See https://github.com/matlab2tikz/matlab2tikz for details
\usepackage{pgfplots}
\pgfplotsset{compat=newest}
%% The following commands are needed for some matlab2tikz features
\usetikzlibrary{plotmarks}
\usetikzlibrary{arrows.meta}
\usepgfplotslibrary{patchplots}
\usepackage{grffile}
%% You may also want the following commands
%\pgfplotsset{plot coordinates/math parser=false}
%\newlength\figureheight
%\newlength\figurewidth

\usepackage[numbib]{tocbibind} % give the references section a section number

\numberwithin{equation}{section} % label equations by sectoins
\numberwithin{figure}{section} % label equations by sectoins

\newenvironment{problem}[2][Problem]{\begin{trivlist}
		\item[\hskip \labelsep {\bfseries #1}\hskip \labelsep {\bfseries #2.}]}{\end{trivlist}}

\begin{document}
	
\title{Robot and Maze Problem}
\author{Charl van de Merwe \\ University of Witwatersrand}
\date{\today}

\maketitle

\section{Introduction}

In this problem the circular robot with two rigid drive wheels, the {\color{blue} \href{https://edu.irobot.com/what-we-offer/create3}{iRobot Create 3}} (see Fig. \ref{fig:iRobot} and Fig. \ref{fig:iRobotBottom}), should run through a maze, Fig. \ref{fig:Maze}, in minimum time.

\begin{figure}[h!]
	\centering
	\begin{minipage}[b]{0.45\textwidth}
		\centering
		\includegraphics[width=7cm]{Figures/iRobot.jpg}
		\caption{iRobot Create 3}
		\label{fig:iRobot}
	\end{minipage}
	\hfill
	\begin{minipage}[b]{0.45\textwidth}
		\centering
		\includegraphics[width=7cm]{Figures/iRobotBottom.png}
		\caption{Bottom of the robot.}
		\label{fig:iRobotBottom}
	\end{minipage}
\end{figure}

\begin{figure}[h!]
	\centering
	\includegraphics[width=8cm]{Figures/Maze.png}
	\caption{The maze from the 2018 APEC Micromouse competition. Each cell is 1m by 1m.}
	\label{fig:Maze}
\end{figure}

I was introduced to the problem my a friend at Wits, who is solving this problem with reinforcement learning. He will solve it in simulation, using the {\color{blue} \href{https://github.com/Yurof/WheeledRobotSimulations}{WheeledRobotSimulations}} GitHub repository. This repository uses the {\color{blue} \href{https://github.com/bulletphysics/bullet3}{PyBullet}} library for the robot dynamics.

\section{Model}

I have not found the details of the exact model used in PyBullet in any of the documentation found on its {\color{blue} \href{https://pybullet.org/wordpress/}{website}}. From what I've gathered, it doesn't seem like they do any complex tyre modelling.

Assuming zero slip (very high adhesion conditions or low speed with high normal forces), the torque of the motor is directly translated to adhesion force:
\begin{equation}\label{eq:Force}
	F = T / r,
\end{equation}
where $F$ is the adhesion force, $T$ is the motor torque, and $r$ is the wheel radius.

The longitudinal dynamics of the robot are:
\begin{equation}
	m \, \dot v = F_R + F_L,
\end{equation}
where $m$ is the mass of the robot, $v$ is the longitudinal velocity, and $F_R$ and $F_L$ are the right and left wheel adhesion forces.

The rotational dynamics of the robot are:
\begin{equation}
	I \, \dot \omega = w \, (F_R - F_L),
\end{equation}
where $I$ is the robot inertia, $\omega$ is the angular velocity, and $w$ is the distance from the center of gravity (CoG) to a wheel (it is assumed the CoG is in the middle of the robot).

The position of the robot is tracked by adding the the following to the model:
\begin{align}
	\dot x &= v \cos \theta \\
	\dot y &= v \sin \theta,
\end{align}
where $x$ and $y$ are the x and y locations of the robot, and $\theta$ is the robot angle relative to the x-axis.

The state vector is chosen as:
\begin{equation}
	\boldsymbol{x} = \begin{bmatrix}
		v & \theta & x & y & \omega
	\end{bmatrix}^T,
\end{equation}
Therefore, the model is:
\begin{equation}
	\boldsymbol{\dot x} = \begin{bmatrix}
		\frac{1}{m} (F_R + F_L) \\
		\omega \\
		v \cos \theta \\
		v \sin \theta \\
		\frac{w}{I} \, (F_R - F_L)
	\end{bmatrix},
\end{equation}
where the forces, $F_R$ and $F_L$, are the inputs (controlled variables).

The model parameters values are given in Table \ref{tab:Model}.
\begin{table}[h!]
	\centering
	\begin{tabular}{c c c}
		\hline
		\textbf{Parameter} & \textbf{Description} & \textbf{Value} \\ \hline
		$m$ & Robot mass. & $5.925$ $\mathrm{kg}$ \\ \hline
		$I$ & Robot inertia. & $^* 0.0836$ $\mathrm{kg \, m^2}$ \\ \hline
		$w$ & Distance from wheel to CoG. & $0.13$ $\mathrm{m}$ \\ \hline
		$r$ & Wheel radius. & $0.03$ $\mathrm{m}$ \\ \hline
		$r_\mathrm{Robot}$ & Robot radius. & $0.168$ $\mathrm{m}$ \\ \hline
	\end{tabular}
	\caption{Model parameters. *The inertia is calculated by assuming the robot is a cylinder with a uniform weight distribution. Therefore, $I = 0.5 \, m \, r_\mathrm{Robot}^2$.}
	\label{tab:Model}
\end{table}

\iffalse
\section{Model Validation}

To test the model, a simulation is run with constant forces, with the right force set higher than the left force. The longitudinal velocity should increase at a constant acceleration, and the robot is expected to run in circles.

Please see the GitHub repository, {\color{blue} \href{https://github.com/CharlvdM/RobotAndMaze}{RobotAndMaze}}, I have created containing the MATLAB and Latex code.
\fi

\section{First GPOPS-II Simulation}

Please see the GitHub repository, {\color{blue} \href{https://github.com/CharlvdM/RobotAndMaze}{RobotAndMaze}}, I have created containing the MATLAB and Latex code.

The maze boundaries should be computed as $x$ and $y$ state constraints based on the location of the robot. GPOPS will not be used to solve the maze. Rather, the incorrect routes will be closed, so that the solved maze is like a racetrack with sharp corners.

\subsection{Boundary Conditions}

Before creating a complex simulation with the full track (solved maze), a problem with the following start and final conditions is constructed, where the objective is to reach the final conditions in minimum time:
\begin{equation}
	\begin{matrix}
		v_0 = 0, & \theta_0 = 0, & x_0 = 0, & y_0 = 0, & \omega_0 = 0, \\
		& \theta_f = 0, & x_f = 20, & y_f = 10. &
	\end{matrix} \nonumber
\end{equation}

\subsection{Limits}
\label{sec:Limits}

The velocity limit in the {\color{blue} \href{https://github.com/Yurof/WheeledRobotSimulations}{WheeledRobotSimulations}} project was set to $16.5$ $\mathrm{m/s}$. From the given boundary conditions, the robot isn't expected to ever do more than a full rotation. The position limits are set slightly outside of the boundary conditions. The angular rate is not expected to be very high. Therefore:
\begin{equation}
	\begin{matrix}
		v_\mathrm{min} = -16.5 & \theta_\mathrm{min} = - 2 & x_\mathrm{min} = - 10 & y_\mathrm{min} = - 10 & \omega_\mathrm{min} = - 10 \\
		v_\mathrm{max} = 16.5 & \theta_\mathrm{max} = 2 \pi & x_\mathrm{max} = 25 & y_\mathrm{max} = 20 & \omega_\mathrm{max} = 10
	\end{matrix} \nonumber
\end{equation}

There is a force limit specified as $20$ (without a unit) in the {\color{blue} \href{https://github.com/Yurof/WheeledRobotSimulations}{WheeledRobotSimulations}} project, which is passed to the motor control function of the {\color{blue} \href{https://github.com/bulletphysics/bullet3}{PyBullet}} library. Since a torque of  $20$ $\mathrm{Nm}$ is not realistic, a torque limit of $0.6 \, \mathrm{Nm}$, resulting in a force limit of $20 \, \mathrm{N}$ from (\ref{eq:Force}), i.e.:
\begin{equation}
	\begin{matrix}
		F_{R/L \, \mathrm{min}} = -20 & F_{R/L \, \mathrm{max}} = 20.
	\end{matrix} \nonumber
\end{equation}

The robot is expected to reach this final time in a few seconds. A final time limit of $10$ seconds is chosen, i.e.:
\begin{equation}
	\begin{matrix}
		t_{0 \, \mathrm{min}} = 0 & t_{f \, \mathrm{min}} = 0 \\
		t_{0 \, \mathrm{max}} = 0 & \ t_{f \, \mathrm{max}} = 10
	\end{matrix} \nonumber
\end{equation}

\subsection{Initial Guess}

For this simple problem, a guess for the initial and final time will be given. The final time is chosen to equal the maximum time, which is expected to be much larger than the actual final time, but it should provide an accurate enough sample point for interpolation for this problem. However, for the full maze problem, more intermediate guesses will probably have to be provided, possibly even tracking the full path of the maze.

The time sample guesses are:
\begin{equation}
	\boldsymbol t_\mathrm{guess} = \begin{bmatrix}
		0 \\ t_{f \, \mathrm{max}}
	\end{bmatrix} \nonumber
\end{equation}

Since the final y position is higher than the start and the final angle should be zero, the final angular rate is expected to be a large negative value. Therefore, the state guesses are:
\begin{equation}
	\boldsymbol{x}_\mathrm{guess} = \begin{bmatrix}
		v_0 & \theta_0 & x_0 & x_f & \omega_0 \\
		v_\mathrm{max} & \theta_f & x_0 & y_f & \omega_\mathrm{min}
	\end{bmatrix} \nonumber
\end{equation}

To minimize time it is expected that the wheel forces will be at a maximum throughout most of the solution:
\begin{equation}
	\boldsymbol{u}_\mathrm{guess} = \begin{bmatrix}
		F_{R \, \mathrm{max}} & F_{L \, \mathrm{max}} \\
		F_{R \, \mathrm{max}} & F_{L \, \mathrm{max}}
	\end{bmatrix} \nonumber
\end{equation}

\subsection{Solver Options}

The dynamics of this problem seemed to be similar to the minimum time to climb example GPOPS example. Therefore, I copied the mesh and solver settings from that example.

\subsection{Simulation Results}
\label{sec:SimResults}

The robot reaches the correct final position, as seen in Fig. \ref{fig:States}, and Fig. \ref{fig:Trajectory}.

\begin{figure}[h!]
	\includegraphics{Figures/States.pdf}
	\centering
	\caption{X and Y positions and velocity.}
	\label{fig:States}
\end{figure}

\begin{figure}[h!]
	\includegraphics{Figures/Trajectory.pdf}
	\centering
	\caption{Robot Y vs X trajectory.}
	\label{fig:Trajectory}
\end{figure}

There are quite sudden changes in angle, as seen in Fig. \ref{fig:Angle}, caused by sudden changes in the forces, as seen in Fig. \ref{fig:Forces}. The sudden changes in force might be difficult to replicate practically. Limiting the rate of change of the forces might be useful in future iterations.

\begin{figure}[h!]
	\includegraphics{Figures/Angle.pdf}
	\centering
	\caption{Robot angle (radians).}
	\label{fig:Angle}
\end{figure}

\begin{figure}[h!]
	\includegraphics{Figures/Forces.pdf}
	\centering
	\caption{Right and left wheel forces.}
	\label{fig:Forces}
\end{figure}

\clearpage  % Ensures all figures are processed before the next section starts

\section{Basic Maze Simulation}

As mentioned, path constraints/bounds are needed to implement the maze. A simple maze will be used first, as shown in Fig. \ref{fig:SimpleMaze}.

\begin{figure}[h!]
	\centering
	\includegraphics[width=6cm]{Figures/SimpleMaze.png}
	\caption{Simple maze. Each cell is 1m by 1m, so that this maze width and height is $4 \times 3 \ \mathrm{m}$.}
	\label{fig:SimpleMaze}
\end{figure}

The maze boundary is enforced by setting path constraints in the \textbf{robotAndMazeContinuous.m} file as follows:
\begin{equation} \label{eq:Bounds}
	\boldsymbol{B} = \begin{bmatrix}
		\boldsymbol{x_\mathrm{Left}} & \boldsymbol{x_\mathrm{Right}} & \boldsymbol{y_\mathrm{Bottom}} & \boldsymbol{y_\mathrm{Top}}
	\end{bmatrix},
\end{equation}
where the number of rows in the path bounds $\boldsymbol{B}$ is equal to the number of collocation points, $\boldsymbol{x_\mathrm{Left}}$ \& $\boldsymbol{x_\mathrm{Right}}$ are the distances from the robot to the closest left and right boundaries for each collocation point, and $\boldsymbol{y_\mathrm{Bottom}}$ \& $\boldsymbol{y_\mathrm{Top}}$ are the distances from the robot to the closest bottom and top boundaries for each collocation point.

Given the definition of the path constraints in (\ref{eq:Bounds}), the upper and lower limits thereof, defined in \textbf{robotAndMazeMain.m}, are:
\begin{align}
	B_\mathrm{lower} &= \begin{bmatrix}
		0 & 0 & 0 & 0
	\end{bmatrix} \\
	B_\mathrm{upper} &= \begin{bmatrix}
		x_\mathrm{max} & x_\mathrm{max} & y_\mathrm{max} & y_\mathrm{max}
	\end{bmatrix},
\end{align}
where $x_\mathrm{max}=4$ and $y_\mathrm{max}=3$ for this problem.

To find the distances in (\ref{eq:Bounds}), a function of the following form is created:
\begin{equation}
	\begin{bmatrix}
		\boldsymbol{x_\mathrm{Left}} & \boldsymbol{x_\mathrm{Right}} & \boldsymbol{y_\mathrm{Bottom}} & \boldsymbol{y_\mathrm{Top}}
	\end{bmatrix} = \mathrm{calcCollisionDistances} \left( \begin{matrix}
	\boldsymbol{x}, & \boldsymbol{y}, & \boldsymbol{M_x}, & \boldsymbol{M_y}
	\end{matrix}
	\right), \nonumber
\end{equation}
where $x$ and $y$ are the xy coordinates for the robot at each collocation point, and $M_x$ and $M_y$ are the x and y collision matrices. These matrices have the following form:
\begin{align}
	M_x = \begin{bmatrix}
		1 & 0 & 1 & 0 \\
		1 & 1 & 0 & 1 \\
		1 & 0 & 0 & 0
	\end{bmatrix} \nonumber \\
	M_x = \begin{bmatrix}
		1 & 1 & 1 & 1 \\
		1 & 0 & 0 & 0 \\
		0 & 1 & 1 & 0
		\end{bmatrix} \nonumber
\end{align}
corresponding to the vertical and horizontal wall positions of Fig. \ref{fig:SimpleMaze} respectively.

Currently, I am unable to solve a problem with phase constraints. In addition, I do not believe that I have implemented an optimal wall collision distances calculation.

%\bibliographystyle{IEEEtran}
%\bibliography{References}
	
\end{document}